\documentclass[a4paper, 12pt]{article}
\usepackage{cmap}
\usepackage[12pt]{extsizes}			
\usepackage{mathtext} 				
\usepackage[T2A]{fontenc}			
\usepackage[utf8]{inputenc}			
\usepackage[english,russian]{babel}
\usepackage{setspace}
\singlespacing
\usepackage{amsmath,amsfonts,amssymb,amsthm,mathtools}
\usepackage{fancyhdr}
\usepackage{soulutf8}
\usepackage{euscript}
\usepackage{mathrsfs}
\usepackage{listings}
\pagestyle{fancy}
\usepackage{indentfirst}
\usepackage[top=10mm]{geometry}
\rhead{}
\lhead{}
\renewcommand{\headrulewidth}{0mm}
\usepackage{tocloft}
\renewcommand{\cftsecleader}{\cftdotfill{\cftdotsep}}
\usepackage[dvipsnames]{xcolor}

\lstdefinestyle{mystyle}{ 
	keywordstyle=\color{OliveGreen},
	numberstyle=\tiny\color{Gray},
	stringstyle=\color{BurntOrange},
	basicstyle=\ttfamily\footnotesize,
	breakatwhitespace=false,         
	breaklines=true,                 
	captionpos=b,                    
	keepspaces=true,                 
	numbers=left,                    
	numbersep=5pt,                  
	showspaces=false,                
	showstringspaces=false,
	showtabs=false,                  
	tabsize=2
}

\lstset{style=mystyle}

\begin{document}
\thispagestyle{empty}	
\begin{center}
	Московский авиационный институт
	
	(Национальный исследовательский университет)
	
	Факультет "Информационные технологии и прикладная математика"
	
\end{center}
\vspace{40ex}
\begin{center}
	\textbf{\large{Лабораторная работа №4 по курсу \textquotedblleft Объектно-ориентированное программирование\textquotedblright}}
\end{center}
\vspace{40ex}
\begin{flushright}
	\textit{Студент: } Живалев Е.А.
	
	\vspace{2ex}
	\textit{Группа: } М8О-206Б
	
	\vspace{2ex}
	\textit{Преподаватель: } Журавлев А.А.
	
	\vspace{2ex}
	\textit{Вариант: } 5
	
	\vspace{2ex}
	\textit{Оценка: } \underline{\quad\quad\quad\quad\quad\quad}
	
	 \vspace{2ex}
	\textit{Дата: } \underline{\quad\quad\quad\quad\quad\quad}
	
\end{flushright}

\begin{vfill}
	\begin{center}
		Москва
		
		2019
	\end{center}	
\end{vfill}
\newpage
\section{Исходный код}

Ссылка на github : https://github.com/QElderDelta/oop\_exercise\_04

\vspace{3ex}
\textbf{\large{vertex.hpp}}
\lstinputlisting[language=C++]{vertex.hpp}

\vspace{3ex}
\textbf{\large{rhombus.hpp}}
\lstinputlisting[language=C++]{rhombus.hpp}

\vspace{3ex}
\textbf{\large{pentagon.hpp}}
\lstinputlisting[language=C++]{pentagon.hpp}

\vspace{3ex}
\textbf{\large{hexagon.hpp}}
\lstinputlisting[language=C++]{hexagon.hpp}

\vspace{3ex}
\textbf{\large{templates.hpp}}
\lstinputlisting[language=C++]{templates.hpp}

\vspace{3ex}
\textbf{\large{main.cpp}}
\lstinputlisting[language=C++]{main.cpp}

\vspace{3ex}
\textbf{\large{CMakeLists.txt}}
\lstinputlisting{CMakeLists.txt}

\vspace{3ex}
\textbf{\large{meson.build}}
\lstinputlisting{meson.build}
\newpage
\section{Тестирование}
\vspace{3ex}

\textbf{test\_01.txt}:

Попробуем создать фигуру с координатами (1, 2), (1, 3), (1, 4), (1, 5), которая очевидно не является ромбом, рассчитывая получить сообщение об ошибке. Затем создадим ромб с координатами (-2, 0), (0, 2), (2, 0), (0, -2), площадь которого равна 8, а центр находится в точке (0, 0), а также пятиугольник с координатами (-2.000, 0.000), (-1.000, 1.000), (1.000, 1.000), (2.000, 0.000), (1.000, -1.000), площадь которого равна 5 и шестиугольник с координатами
(-2.000, 0.000), (-1.000, 1.000), (1.000, 1.000), (2.000, 0.000), (1.000, -1.000), (-1.000, -1.000) с площадью равной 6.

Результат:

1 - Rhombus

2 - Pentagon

3 - Hexagon

0 - Exit

Entered coordinates are not forming Rhombus. Try entering new coordinates

Rhombus: [-2.000, 0.000] [0.000, 2.000] [2.000, 0.000] [0.000, -2.000]

8.000

[0.000, 0.000]

Rhombus: [-2.000, 0.000] [0.000, 2.000] [2.000, 0.000] [0.000, -2.000] 

8.000

[0.000, 0.000]
Pentagon: [-2.000, 0.000] [-1.000, 1.000] [1.000, 1.000] [2.000, 0.000] [1.000, -1.000] 

5.000

[0.200, 0.200]

Pentagon: [-2.000, 0.000] [-1.000, 1.000] [1.000, 1.000] [2.000, 0.000] [1.000, -1.000] 

5.000

[0.200, 0.200]

Hexagon: [-2.000, 0.000] [-1.000, 1.000] [1.000, 1.000] [2.000, 0.000] [1.000, -1.000] [-1.000, -1.000] 

6.000

[0.000, 0.000]

Hexagon: [-2.000, 0.000] [-1.000, 1.000] [1.000, 1.000] [2.000, 0.000] [1.000, -1.000] [-1.000, -1.000] 

6.000

[0.000, 0.000]





\vspace{3ex}

\textbf{test\_02.txt} 

Создадим ромб с координатами [4.000, 0.000], [8.000, 2.000], [12.000, 0.000], [8.000, -2.000], центром в точке [8, 0] и площадью равной 16, квадрат с координатами [4.000, 2.000], [8.000, 2.000], [8.000, -2.000], [4.000, -2.000] с центром в точке [6, 0] и площадью равной 16, пятиугольник с координатами [4.000, 0.000], [8.000, 2.000], [12.000, 0.000], [8.000, -2.000], [6.000, -2.000] и площадью равной 18,  шестиугольник с координатами [4.000, 0.000], [8.000, 2.000], [10.000, 2.000], [12.000, 0.000], [8.000, -2.000], [6.000, -2.000] и площадью равной 20.

Результат:

1 - Rhombus

2 - Pentagon

3 - Hexagon

0 - Exit

Rhombus: [4.000, 0.000] [8.000, 2.000] [12.000, 0.000] [8.000, -2.000] 

16.000

[8.000, 0.000]

Rhombus: [4.000, 0.000] [8.000, 2.000] [12.000, 0.000] [8.000, -2.000] 

16.000

[8.000, 0.000]

Rhombus: [4.000, 2.000] [8.000, 2.000] [8.000, -2.000] [4.000, -2.000] 

16.000

[6.000, 0.000]

Rhombus: [4.000, 2.000] [8.000, 2.000] [8.000, -2.000] [4.000, -2.000] 

16.000

[6.000, 0.000]

Pentagon: [4.000, 0.000] [8.000, 2.000] [12.000, 0.000] [8.000, -2.000] [6.000, -2.000] 

18.000

[7.600, -0.400]

Pentagon: [4.000, 0.000] [8.000, 2.000] [12.000, 0.000] [8.000, -2.000] [6.000, -2.000] 

18.000

[7.600, -0.400]

Hexagon: [4.000, 0.000] [8.000, 2.000] [10.000, 2.000] [12.000, 0.000] [8.000, -2.000] [6.000, -2.000] 

20.000

[8.000, 0.000]

Hexagon: [4.000, 0.000] [8.000, 2.000] [10.000, 2.000] [12.000, 0.000] [8.000, -2.000] [6.000, -2.000] 

20.000

[8.000, 0.000]

\newpage

\section{Объяснение результатов работы программы}

При вводе координат для создания ромба производится проверка этих координат, ведь они могут не образовывать ромб. Для этого реализована функция checkIfRhombus, которая вычисляет расстояния от одной точки до трёх остальных, а поскольку фигура является ромбом, то два из низ должны быть равны. Третье же значение функция возвращает ведь оно равно длине одной из диагоналей. Площадь ромба вычисляется как половина произведения диагоналей, центр - точка пересечения диагоналей. Методы вычисления площади и центра для пяти- и шестиугольника совпадают. Чтобы найти площадь необходимо перебрать все ребра и сложить площади трапеций, ограниченных этими ребрами. Чтобы найти центр необходимо разбить фигуры на треугольники(найти одну точку внутри фигуры), для каждого треугольника найти центроид и площадь и перемножить их, просуммировать полученные величины и разделить на общую площадь фигуры.   

\newpage
\section{Выводы}

На мой взгляд, метапрограммирование очень хорошо развито в плюсах. Я на своем примере увидел насколько меньше можно написать кода, если использовать предлагаемые для этого языком механизмы. Также я познакомился с системой для автоматизации сборки meson. По-моему, она имеет более приятный синтаксис, чем cmake, и, в большинстве, из-за этого она мне нравится больше.
\end{document}